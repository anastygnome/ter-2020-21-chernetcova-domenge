\usepackage{cancel}
%%%%%%%%%%%%%%%%%%%%%%%%%%%%%
%   THEOREMES SANS BOITES   %
%%%%%%%%%%%%%%%%%%%%%%%%%%%%%
\theoremstyle{break}
\theoremseparator{~:} % espace fine insécable avant le :
\newtheorem{lemma}{Lemme}
\newtheorem{corollary}{Corollaire}
\newtheorem{definition}{Définition}
\theoremstyle{plain}
\newtheorem*{question}{Question}
\newtheorem*{answer}{Réponse}
\newtheorem{remark}{Remarque}
\theoremsymbol{\text{\textsc{c.q.f.d}}} % mod
\theorembodyfont{\normalfont}
\theoremprework{\setcounter{proofpart}{0}}
\newtheorem*{proof}{Démonstration}
%%%%%%%%%%%%%%%%%%%%%%%%%%%%%
%            COULEURS       %
%%%%%%%%%%%%%%%%%%%%%%%%%%%%%
\definecolor{vert}{RGB}{0,181,0}
\definecolor{oran}{RGB}{223,74,0}
\definecolor{viol}{RGB}{134,0,175}
\definecolor{roug}{RGB}{215,15,0}
\definecolor{bleu}{RGB}{0,104,180}

%%%%%%%%%%%%%%%%%%%%%%%%%%%%%
%   BOITES POUR THEOREMES   %
%%%%%%%%%%%%%%%%%%%%%%%%%%%%%
\tcbsetforeverylayer{shield externalize}% <--- interim solution before bug fix
\tcbset{separator sign={},
        description delimiters parenthesis,
        label separator=:,
styletheorem/.style={enhanced,
  boxrule=\fboxrule,
  coltitle=black,
  colback=white,
  before skip=2\fboxsep,
  after skip=2\fboxsep,
  fonttitle=\bfseries,
  boxed title style={boxrule=\fboxrule},
  attach boxed title to top left={yshift=-2mm, xshift=2mm},
    }%
}
\newtcbtheorem[auto counter, number within = section]
{theoremb}{Théorème}{styletheorem,colframe=roug,
colback=white!90!roug,colbacktitle=white!80!roug,label type=theorem}{thm}
\newtcbtheorem[auto counter, number within = section]
{remarkb}{Remarque}{styletheorem,colframe=oran,
colbacktitle=white!80!oran,colback=white!90!oran,label type=remark}{rem}
\newtcbtheorem[auto counter, number within = section]
{defb}{Définition}{styletheorem,colframe=bleu,
colbacktitle=white!80!bleu,colback=white!90!bleu,label type=definition}{def}
\newtcbtheorem[auto counter, number within = section]
{noteb}{Commentaire}{styletheorem,colframe=vert,
colbacktitle=white!80!vert,colback=white!90!vert,label type=note}{note}
\newtcbtheorem[auto counter, number within = section]
{lemmab}{Lemme}{styletheorem,colframe=roug,
colback=white!90!roug,colbacktitle=white!80!roug,label type=lemma}{lem}
% le label type fait automatiquement la jonction avec cleveref pour nommer les théorèmes : le dernier groupe (thm,rem,def) permet de créer des labels automatiquement. 

%%%%%%%%%%%%%%%%%%%%%%%%%%%%%%%%%%
%  SEPARATEUR (DANS LES PREUVES) : \proofpart   %
%%%%%%%%%%%%%%%%%%%%%%%%%%%%%%%%%%%
% une  commande est définie pour séparer les preuves en deux variantes : avec et sans étoiles. (\proofpart et \proofpart*)
% Par défaut, elle crée des sous parties dans un environement de démonstration sous la forme 
% " Partie n : [titre en italique]. ", où n est un entier naturel strictement positif. Dans le cas ou le titre est vide, le point (.) n'est pas ajouté. Si le titre est vide, il faut utiliser \proofpart{}
% Avec une étoile, on obtient 
%[titre en italique : ] 

\newcommand{\deffunct}[5]{%
\begin{align*}%
      #1 \colon & #2 \to #3\\
       &#4\xmapsto{\hphantom{#1}} #5
\end{align*}%
}


%%%%%%%%%%%%%%%%%%%%%%%%%%%%%%%%%%%%%%%%%%%%%%
%               Ensembles.                   %
%%%%%%%%%%%%%%%%%%%%%%%%%%%%%%%%%%%%%%%%%%%%%%
\newcommand*{\NN}{\mathbf{N}}
\newcommand*{\ZZ}{\mathbf{Z}}
\newcommand*{\QQ}{\mathbf{Q}}
%%%%%%%%%%%%%%%%%%%%%%%%%%%%%
%   PAGE DE GARDE            %
%%%%%%%%%%%%%%%%%%%%%%%%%%%%%
\newcommand{\HRule}{\rule{\paperwidth}{0.5mm}} % trait, régler eppaisseur
\newcommand*{\theuniversity}{Université de Toulon}
\newcommand*{\theyearname}{Licence de Mathématiques, parcours mathématiques, 
2\ieme~année}
\newcommand*{\thesupervisor}{Joachim \bsc{Asch}}
\author{Anastasiia \bsc{Chernetcova}~\&~Tom \bsc{Domenge}}
\title{Projet TER 2020\par
            Dénombrabilité : \par $\NN,2\NN \ \& \ \QQ$\par}


%%%%%%%%%%%%%%%%%%%%%%%%%%%%%%%%%%%%%%%%%%%%%%
%   	Marges et note en marge.   			 %
%%%%%%%%%%%%%%%%%%%%%%%%%%%%%%%%%%%%%%%%%%%%%%
\setulmargins{*}{*}{*}
\setheaderspaces{*}{*}{*}
\setlrmargins{*}{*}{*}
\setheadfoot{\headheight+0.314pt}{\footskip}
\checkandfixthelayout[nearest]
\renewcommand*{\sideparfont}{\itshape\footnotesize}
\renewcommand*{\sideparform}{\ifmemtortm\RaggedRight\else\RaggedLeft\fi}

%%%%%%%%%%%%%%%%%%%%%%%%%%%%%%%%%%%%%%%%%%%%%%%%
%            Numérotation (classe memoir)       %
%%%%%%%%%%%%%%%%%%%%%%%%%%%%%%%%%%%%%%%%%%%%%%%%%

\setsecnumdepth{subsubsection}
\renewcommand{\cftpartaftersnum}{.}
\renewcommand{\cftchapteraftersnum}{.}
\renewcommand{\cftpartdotsep}{\cftdotsep}
\renewcommand{\cftchapterdotsep}{\cftdotsep}% Chapters should use dots in ToC
\aliaspagestyle{title}{empty} 
\aliaspagestyle{part}{empty} 
\addto\captionsfrench{\renewcommand{\appendixpagename}{Annexes}}
\addto\captionsfrench{\renewcommand{\appendixtocname}{Annexes}}

%%%%%%%%%%%%%%%%%%%%%%%%%%%%%%%%%%%%%%%%%%%%%%%%
%            Numérotation (classe memoir)       %
%%%%%%%%%%%%%%%%%%%%%%%%%%%%%%%%%%%%%%%%%%%%%%%%%
\setmainfont{CMU Serif}
\setsansfont{CMU Sans Serif}
\setmonofont{CMU Typewriter Text}
\setmathfont{Asana Math}
\setmathfont[range={bb}]{STIX Two Math}
\setmathfont[range={scr,cal}]{TeX Gyre Pagella Math}

%%%%%%%%%%%%%%%%%%%%%%%%%%%%%%%%%%%%%%%%%%%%%%
%   Opérateurs (dans le style de max etc.   %
%%%%%%%%%%%%%%%%%%%%%%%%%%%%%%%%%%%%%%%%%%%%%%
\DeclareMathOperator{\Card}{Card} % s'utilise avec \Card en mode maths
