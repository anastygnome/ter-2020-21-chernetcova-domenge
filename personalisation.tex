\usepackage{cancel}
%%%%%%%%%%%%%%%%%%%%%%%%%%%%%
%   THEOREMES SANS BOITES   %
%%%%%%%%%%%%%%%%%%%%%%%%%%%%%
\theoremstyle{break}
\theoremseparator{~:} % espace fine insécable avant le :
\newtheorem{lemma}{Lemme}
\newtheorem{corollary}{Corollaire}
\newtheorem{definition}{Définition}
\theoremstyle{plain}
\newtheorem*{question}{Question}
\newtheorem*{answer}{Réponse}
\newtheorem{remark}{Remarque}
\theoremsymbol{\text{\textsc{c.q.f.d}}} % mod
\theorembodyfont{\normalfont}
\theoremprework{\setcounter{proofpart}{0}}
\newtheorem*{proof}{Démonstration}
%%%%%%%%%%%%%%%%%%%%%%%%%%%%%
%            COULEURS       %
%%%%%%%%%%%%%%%%%%%%%%%%%%%%%
\definecolor{vert}{RGB}{0,181,0}
\definecolor{oran}{RGB}{223,74,0}
\definecolor{viol}{RGB}{134,0,175}
\definecolor{roug}{RGB}{215,15,0}
\definecolor{bleu}{RGB}{0,104,180}

%%%%%%%%%%%%%%%%%%%%%%%%%%%%%
%   BOITES POUR THEOREMES   %
%%%%%%%%%%%%%%%%%%%%%%%%%%%%%
\tcbsetforeverylayer{shield externalize}% <--- interim solution before bug fix
\tcbset{separator sign={},
        description delimiters parenthesis,
        label separator=:,
styletheorem/.style={enhanced,
  boxrule=\fboxrule,
  coltitle=black,
  colback=white,
  before skip=2\fboxsep+\parskip,
  after skip=2\fboxsep+\parskip,
  fonttitle=\bfseries,
  boxed title style={boxrule=\fboxrule},
  attach boxed title to top left={yshift=-2mm, xshift=2mm},
    }%
}
\newtcbtheorem[auto counter, number within = section]
{theoremb}{Théorème}{styletheorem,colframe=roug,
colback=white!90!roug,colbacktitle=white!80!roug,label type=theorem}{thm}
\newtcbtheorem[auto counter, number within = section]
{remarkb}{Remarque}{styletheorem,colframe=oran,
colbacktitle=white!80!oran,colback=white!90!oran,label type=remark}{rem}
\newtcbtheorem[auto counter, number within = section]
{defb}{Définition}{styletheorem,colframe=bleu,
colbacktitle=white!80!bleu,colback=white!90!bleu,label type=definition}{def}
\newtcbtheorem[auto counter, number within = section]
{noteb}{Commentaire}{styletheorem,colframe=vert,
colbacktitle=white!80!vert,colback=white!90!vert,label type=note}{note}
\newtcbtheorem[auto counter, number within = section]
{lemmab}{Lemme}{styletheorem,colframe=roug,
colback=white!90!roug,colbacktitle=white!80!roug,label type=lemma}{lem}
% le label type fait automatiquement la jonction avec cleveref pour nommer les théorèmes : le dernier groupe (thm,rem,def) permet de créer des labels automatiquement. 

%%%%%%%%%%%%%%%%%%%%%%%%%%%%%%%%%%
%  SEPARATEUR (DANS LES PREUVES) : \proofpart   %
%%%%%%%%%%%%%%%%%%%%%%%%%%%%%%%%%%%
% une  commande est définie pour séparer les preuves en deux variantes : avec et sans étoiles. (\proofpart et \proofpart*)
% Par défaut, elle crée des sous parties dans un environement de démonstration sous la forme 
% " Partie n : [titre en italique]. ", où n est un entier naturel strictement positif. Dans le cas ou le titre est vide, le point (.) n'est pas ajouté. Si le titre est vide, il faut utiliser \proofpart{}
% Avec une étoile, on obtient 
%[titre en italique : ] 

\newcommand{\deffunct}[5]{%
\begin{align*}%
      #1 \colon & #2 \to #3\\
       &#4\xmapsto{\hphantom{#1}} #5
\end{align*}%
}


%%%%%%%%%%%%%%%%%%%%%%%%%%%%%%%%%%%%%%%%%%%%%%
%               Ensembles.                   %
%%%%%%%%%%%%%%%%%%%%%%%%%%%%%%%%%%%%%%%%%%%%%%
\newcommand*{\NN}{\mathbf{N}}
\newcommand*{\ZZ}{\mathbf{Z}}
\newcommand*{\QQ}{\mathbf{Q}}
%%%%%%%%%%%%%%%%%%%%%%%%%%%%%
%   PAGE DE GARDE            %
%%%%%%%%%%%%%%%%%%%%%%%%%%%%%
\newcommand{\HRule}{\rule{\paperwidth}{0.5mm}} % trait, régler eppaisseur
\newcommand*{\theuniversity}{Université de Toulon}
\newcommand*{\theyearname}{Licence de Mathématiques, parcours mathématiques, 
2\ieme~année}
\newcommand*{\thesupervisor}{Joachim \bsc{Asch}}
\author{Anastasiia \bsc{Chernetcova}~\&~Tom \bsc{Domenge}}
\title{Projet TER 2020\par
            Dénombrabilité : \par $\NN,2\NN \text{ \bfseries\& } \QQ$\par}


%%%%%%%%%%%%%%%%%%%%%%%%%%%%%%%%%%%%%%%%%%%%%%
%   	Marges et note en marge.   			 %
%%%%%%%%%%%%%%%%%%%%%%%%%%%%%%%%%%%%%%%%%%%%%%
\setulmargins{*}{*}{*}
\setheaderspaces{*}{*}{*}
\setlrmargins{*}{*}{*}
\setheadfoot{\headheight}{\footskip}
\checkandfixthelayout[nearest]
\renewcommand*{\sideparfont}{\itshape\footnotesize}
\renewcommand*{\sideparform}{\ifmemtortm\RaggedRight\else\RaggedLeft\fi}

%%%%%%%%%%%%%%%%%%%%%%%%%%%%%%%%%%%%%%%%%%%%%%%%
%            Numérotation (classe memoir)       %
%%%%%%%%%%%%%%%%%%%%%%%%%%%%%%%%%%%%%%%%%%%%%%%%

\setsecnumdepth{subsubsection}
\renewcommand{\cftpartaftersnum}{.}
\renewcommand{\cftchapteraftersnum}{.}
\renewcommand{\cftpartdotsep}{\cftdotsep}
\renewcommand{\cftchapterdotsep}{\cftdotsep}% Chapters should use dots in ToC
\aliaspagestyle{title}{empty}
\aliaspagestyle{part}{empty}
\addto\captionsfrench{\renewcommand{\appendixpagename}{Annexes}}
\addto\captionsfrench{\renewcommand{\appendixtocname}{Annexes}}

%%%%%%%%%%%%%%%%%%%%%%%%%%%%%%%%%%%%%%%%%%%%%%%%
%            Polices                           %
%%%%%%%%%%%%%%%%%%%%%%%%%%%%%%%%%%%%%%%%%%%%%%%%
\setmainfont{TeX Gyre Pagella}
\setmathfont{Asana Math}
\setmathfont[range={"0000-"0001,"0020-"007E,
                    "00A0,"00A7-"00A8,"00AC,"00AF,"00B1,"00B4-"00B5,"00B7,
                    "00D7,"00F7,
                    "0131,
                    "0237,"02C6-"02C7,"02D8-"02DA,"02DC,
                    "0300-"030C,"030F,"0311,"0323-"0325,"032E-"0332,"0338,
                    "0391-"0393,"0395-"03A1,"03A3-"03A8,"03B1-"03BB,
                    "03BD-"03C1,"03C3-"03C9,"03D1,"03D5-"03D6,"03F5,
                    "2016,"2018-"2019,"2021,"2026-"202C,"2032-"2037,"2044,
                    "2057,"20D6-"20D7,"20DB-"20DD,"20E1,"20EE-"20EF,
                    "210B-"210C,"210E-"2113,"2118,"211B-"211C,"2126-"2128,
                    "212C-"212D,"2130-"2131,"2133,"2135,"2190-"2199,
                    "21A4,"21A6,"21A9-"21AA,"21BC-"21CC,"21D0-"21D5,
                    "2200,"2202-"2209,"220B-"220C,"220F-"2213,"2215-"221E,
                    "2223,"2225,"2227-"222E,"2234-"2235,"2237-"223D,
                    "2240-"224C,"2260-"2269,"226E-"2279,"2282-"228B,"228E,
                    "2291-"2292,"2295-"2299,"22A2-"22A5,"22C0-"22C5,
                    "22DC-"22DD,"22EF,"22F0-"22F1,
                    "2308-"230B,"2320-"2321,"2329-"232A,"239B-"23AE,
                    "23DC-"23DF,
                    "27E8-"27E9,"27F5-"27FE,"2A0C,"2B1A,
                    "1D400-"1D433,"1D49C,"1D49E-"1D49F,"1D4A2,"1D4A5-"1D4A6,
                    "1D4A9-"1D4AC,"1D4AE-"1D4B5,"1D4D0-"1D4E9,"1D504-"1D505,
                    "1D507-"1D50A,"1D50D-"1D514,"1D516-"1D51C,"1D51E-"1D537,
                    "1D56C-"1D59F,"1D6A8-"1D6B8,"1D6BA-"1D6D2,"1D6D4-"1D6DD,
                    "1D6DF,"1D6E1,"1D7CE-"1D7D7
                   }]{euler.otf}
\setmathfont[range=up/{greek,Greek}, script-features={}, sscript-features={}
            ]{euler.otf}
\setmathfont[range=up/{latin,Latin}, script-features={}, sscript-features={}
            ]{euler.otf}
\setmathfont[range={bfup/{latin, Latin, greek, Greek}, frak, bffrak},
             script-features={}, sscript-features={}
            ]{euler.otf}
\setmathfont[range={up/num, bfup/num, it, bfit,
                    sfup, sfit, bfsfup, bfsfit, tt}
                  ]{Asana Math}
\setmathfont[range={bb},Scale=MatchUppercase]{STIX Two Math}
\setmathfont[range={scr,cal},Scale=MatchUppercase]{TeX Gyre Pagella Math}
 \setmathfont[range=bfcal, Scale=MatchUppercase, Alternate]{Asana Math}
%%%%%%%%%%%%%%%%%%%%%%%%%%%%%%%%%%%%%%%%%%%%%%
%   Opérateurs (dans le style de max etc.   %
%%%%%%%%%%%%%%%%%%%%%%%%%%%%%%%%%%%%%%%%%%%%%%
\DeclareMathOperator{\Card}{Card} % s'utilise avec \Card en mode maths
\newcommand{\square}{\mdlgwhtsquare}
