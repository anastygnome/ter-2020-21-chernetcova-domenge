\usepackage{luacode}
\usepackage{shellesc}
\usepackage[german,main=french]{babel}
\usepackage[autostyle=true,maxlevel=3]{csquotes}
\usepackage{fontspec}
\usepackage{geometry}
\usepackage{mathtools} 
\usepackage[math-style=french,warnings-off={mathtools-colon,mathtools-overbracket}]{unicode-math}
\defaultfontfeatures{Ligatures={TeX},Scale={MatchLowercase}}
\usepackage[amsmath,thmmarks,hyperref]{ntheorem}
\usepackage[most]{tcolorbox}
\usepackage{ragged2e}  % Permet la césure dans les marges
\usepackage{tikz}
\usetikzlibrary{calc,trees,positioning,arrows,fit,shapes,calc}
\usepackage{caption}
\usepackage{float}
\usepackage[section]{placeins}
\usepackage{lualatex-math}
%\usepackage{lua-typo}
\usepackage[unicode,naturalnames]{hyperref}
\usepackage{cleveref}
\AtBeginDocument{\let\savedemptyset\emptyset} % 
\AtBeginDocument{\let\emptyset\varnothing} % 
\AtBeginDocument{\let\savedphi\phi} % 
\AtBeginDocument{\let\phi\varphi} %
\AtBeginDocument{\let\varphi\savedphi} % 
\AtBeginDocument{\let\savedleq\leq} % 
\AtBeginDocument{\let\savedgeq\geq} %
\AtBeginDocument{\let\leq\leqslant} % 
\AtBeginDocument{\let\geq\geqslant} %
\newcommand{\paral}{\mathrel{\!/\mkern-5mu/\!}} % \parallel existe déjà : || vs //
\makeatletter
\newcommand*{\theauthor}{\@author}
\newcommand*{\thetitle}{\@title}
\newcommand*{\thedate}{\@date}
\renewcommand{\xmapsto}[2][]{\mathrel{\mathpalette\xmapsto@{{#1}{#2}}}}
\renewcommand{\xmapsto}[2][]{\mathrel{\mathpalette\xmapsto@{{#1}{#2}}}}
\newcommand{\xmapsto@}[2]{\xmapsto@@{#1}#2}
\newcommand{\xmapsto@@}[3]{%
  \begingroup
  \sbox\z@{$\m@th#1\mathop{}\limits_{\;#2\;}^{\;#3\;}$}%
  \mathop{\Uhextensible width \wd\z@ 0 "27FC}_{#2}^{#3}%
  \endgroup
}
\renewcommand{\xLeftrightarrow}[2][]{\mathrel{\mathpalette\xLeftrightarrow@{{#1}{#2}}}}
\newcommand{\xLeftrightarrow@}[2]{\xLeftrightarrow@@{#1}#2}
\newcommand{\xLeftrightarrow@@}[3]{%
  \begingroup
  \sbox\z@{$\m@th#1\mathop{}\limits_{\;#2\;}^{\;#3\;}$}%
  \mathop{\Uhextensible width \wd\z@ 0 "027FA}_{#2}^{#3}% 027FA code for long left right double arrow in unicode-math-table.tex 
  \endgroup
}

\newcounter{proofpart}
\newcommand{\proofpart}{\@ifstar{\@proofpart}{\@@proofpart}}

\newcommand{\@proofpart}[1]{%
  \if\detokenize{#1}\relax\else{
  \par
  \addvspace{\medskipamount}%
  \noindent \itshape%
  {#1~:\par\nobreak\smallskip}%
  \normalfont
  \@afterheading
}\fi
}

\newcommand{\@@proofpart}[1]{%
  \par
  \addvspace{\medskipamount}%
  \stepcounter{proofpart}%
  \noindent Partie \theproofpart~:~\itshape%
  \if\detokenize{#1}\relax%
  \else{#1.}\fi%
  \par\nobreak\smallskip
  \normalfont
  \@afterheading
}
% 2 lignes pour localiser les overfull hbox et vbox dans le log
\showboxdepth=\maxdimen
\showboxbreadth=\maxdimen
\newcounter{PageNumberBeforeMainmatter}
\let\LaTeXStandardMainmatter\mainmatter%
\renewcommand{\mainmatter}{%
\setcounter{PageNumberBeforeMainmatter}{\number\value{page}}%
\LaTeXStandardMainmatter%
}%

\let\LaTeXStandardAppendixPage\appendixpage
\renewcommand{\appendixpage}{
\openany
  \LaTeXStandardAppendixPage%
\pagenumbering{roman}%
\setcounter{page}{\number\value{PageNumberBeforeMainmatter}+1}
}%
