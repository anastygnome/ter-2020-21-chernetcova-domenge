%+ !TeX encoding = UTF-8
% !TeX spellcheck = fr-FR
\documentclass[a4paper,french,final]{memoir}
\usepackage{luacode}
\usepackage{shellesc}
\usepackage[german,main=french]{babel}
\usepackage[autostyle=true,maxlevel=3]{csquotes}
\usepackage{fontspec}
\usepackage{geometry}
\usepackage{mathtools} 
\usepackage[math-style=french,warnings-off={mathtools-colon,mathtools-overbracket}]{unicode-math}
\defaultfontfeatures{Ligatures={TeX}}
\usepackage[amsmath,thmmarks,hyperref]{ntheorem}
\usepackage[most]{tcolorbox}
\usepackage{ragged2e}  % Permet la césure dans les marges
\usepackage{tikz}
\usetikzlibrary{calc,trees,positioning,arrows,fit,shapes,calc}
\usepackage{caption}
\usepackage{lualatex-math}
\usepackage{float}
\usepackage[unicode,naturalnames]{hyperref}
\usepackage{cleveref}
\AtBeginDocument{\let\savedemptyset\emptyset} % 
\AtBeginDocument{\let\emptyset\varnothing} % 
\AtBeginDocument{\let\savedphi\phi} % 
\AtBeginDocument{\let\phi\varphi} %
\AtBeginDocument{\let\varphi\savedphi} % 
\AtBeginDocument{\let\savedleq\leq} % 
\AtBeginDocument{\let\savedgeq\geq} %
\AtBeginDocument{\let\leq\leqslant} % 
\AtBeginDocument{\let\geq\geqslant} %
\newcommand{\paral}{\mathrel{\!/\mkern-5mu/\!}} % \parallel existe déjà : || vs //
\makeatletter
\newcommand*{\theauthor}{\@author}
\newcommand*{\thetitle}{\@title}
\newcommand*{\thedate}{\@date}
\renewcommand{\xmapsto}[2][]{\mathrel{\mathpalette\xmapsto@{{#1}{#2}}}}
\renewcommand{\xmapsto}[2][]{\mathrel{\mathpalette\xmapsto@{{#1}{#2}}}}
\newcommand{\xmapsto@}[2]{\xmapsto@@{#1}#2}
\newcommand{\xmapsto@@}[3]{%
  \begingroup
  \sbox\z@{$\m@th#1\mathop{}\limits_{\;#2\;}^{\;#3\;}$}%
  \mathop{\Uhextensible width \wd\z@ 0 "27FC}_{#2}^{#3}%
  \endgroup
}
\renewcommand{\xLeftrightarrow}[2][]{\mathrel{\mathpalette\xLeftrightarrow@{{#1}{#2}}}}
\newcommand{\xLeftrightarrow@}[2]{\xLeftrightarrow@@{#1}#2}
\newcommand{\xLeftrightarrow@@}[3]{%
  \begingroup
  \sbox\z@{$\m@th#1\mathop{}\limits_{\;#2\;}^{\;#3\;}$}%
  \mathop{\Uhextensible width \wd\z@ 0 "027FA}_{#2}^{#3}% 027FA code for long left right double arrow in unicode-math-table.tex 
  \endgroup
}

\newcounter{proofpart}
\newcommand{\proofpart}{\@ifstar{\@proofpart}{\@@proofpart}}

\newcommand{\@proofpart}[1]{%
  \if\detokenize{#1}\relax\else{
  \par
  \addvspace{\medskipamount}%
  \noindent \itshape%
  {#1~:\par\nobreak\smallskip}%
  \normalfont
  \@afterheading
}\fi
}

\newcommand{\@@proofpart}[1]{%
  \par
  \addvspace{\medskipamount}%
  \stepcounter{proofpart}%
  \noindent Partie \theproofpart~:~\itshape%
  \if\detokenize{#1}\relax%
  \else{#1.}\fi%
  \par\nobreak\smallskip
  \normalfont
  \@afterheading
}
% 2 lignes pour localiser les overfull hbox et vbox dans le log
\showboxdepth=\maxdimen
\showboxbreadth=\maxdimen
\newcounter{PageNumberBeforeMainmatter}
\let\LaTeXStandardMainmatter\mainmatter%
\renewcommand{\mainmatter}{%
\setcounter{PageNumberBeforeMainmatter}{\number\value{page}}%
\LaTeXStandardMainmatter%
}%

\let\LaTeXStandardAppendixPage\appendixpage
\renewcommand{\appendixpage}{%
\cleardoublepage%
\pagenumbering{roman}%
\setcounter{page}{\number\value{PageNumberBeforeMainmatter}+1}
\LaTeXStandardAppendixPage%
}%


\usepackage[backend=biber,style=alphabetic]{biblatex}
\addbibresource{references.bib}
\usepackage{cancel}
%%%%%%%%%%%%%%%%%%%%%%%%%%%%%
%   THEOREMES SANS BOITES   %
%%%%%%%%%%%%%%%%%%%%%%%%%%%%%
\theoremstyle{break}
\theoremseparator{~:} % espace fine insécable avant le :
\newtheorem{lemma}{Lemme}
\newtheorem{corollary}{Corollaire}
\newtheorem{definition}{Définition}
\theoremstyle{plain}
\newtheorem*{question}{Question}
\newtheorem*{answer}{Réponse}
\newtheorem{remark}{Remarque}
\theoremsymbol{\text{\textsc{c.q.f.d}}} % mod
\theorembodyfont{\normalfont}
\theoremprework{\setcounter{proofpart}{0}}
\newtheorem*{proof}{Démonstration}
%%%%%%%%%%%%%%%%%%%%%%%%%%%%%
%            COULEURS       %
%%%%%%%%%%%%%%%%%%%%%%%%%%%%%
\definecolor{vert}{RGB}{0,181,0}
\definecolor{oran}{RGB}{223,74,0}
\definecolor{viol}{RGB}{134,0,175}
\definecolor{roug}{RGB}{215,15,0}
\definecolor{bleu}{RGB}{0,104,180}

%%%%%%%%%%%%%%%%%%%%%%%%%%%%%
%   BOITES POUR THEOREMES   %
%%%%%%%%%%%%%%%%%%%%%%%%%%%%%
\tcbsetforeverylayer{shield externalize}% <--- interim solution before bug fix
\tcbset{separator sign={},
        description delimiters parenthesis,
        label separator=:,
styletheorem/.style={enhanced,
  boxrule=\fboxrule,
  coltitle=black,
  colback=white,
  before skip=2\fboxsep+\parskip,
  after skip=2\fboxsep+\parskip,
  fonttitle=\bfseries,
  boxed title style={boxrule=\fboxrule},
  attach boxed title to top left={yshift=-2mm, xshift=2mm},
    }%
}
\newtcbtheorem[auto counter, number within = section]
{theoremb}{Théorème}{styletheorem,colframe=roug,
colback=white!90!roug,colbacktitle=white!80!roug,label type=theorem}{thm}
\newtcbtheorem[auto counter, number within = section]
{remarkb}{Remarque}{styletheorem,colframe=oran,
colbacktitle=white!80!oran,colback=white!90!oran,label type=remark}{rem}
\newtcbtheorem[auto counter, number within = section]
{defb}{Définition}{styletheorem,colframe=bleu,
colbacktitle=white!80!bleu,colback=white!90!bleu,label type=definition}{def}
\newtcbtheorem[auto counter, number within = section]
{noteb}{Commentaire}{styletheorem,colframe=vert,
colbacktitle=white!80!vert,colback=white!90!vert,label type=note}{note}
\newtcbtheorem[auto counter, number within = section]
{lemmab}{Lemme}{styletheorem,colframe=roug,
colback=white!90!roug,colbacktitle=white!80!roug,label type=lemma}{lem}
% le label type fait automatiquement la jonction avec cleveref pour nommer les théorèmes : le dernier groupe (thm,rem,def) permet de créer des labels automatiquement. 

%%%%%%%%%%%%%%%%%%%%%%%%%%%%%%%%%%
%  SEPARATEUR (DANS LES PREUVES) : \proofpart   %
%%%%%%%%%%%%%%%%%%%%%%%%%%%%%%%%%%%
% une  commande est définie pour séparer les preuves en deux variantes : avec et sans étoiles. (\proofpart et \proofpart*)
% Par défaut, elle crée des sous parties dans un environement de démonstration sous la forme 
% " Partie n : [titre en italique]. ", où n est un entier naturel strictement positif. Dans le cas ou le titre est vide, le point (.) n'est pas ajouté. Si le titre est vide, il faut utiliser \proofpart{}
% Avec une étoile, on obtient 
%[titre en italique : ] 

\newcommand{\deffunct}[5]{%
\begin{align*}%
      #1 \colon & #2 \to #3\\
       &#4\xmapsto{\hphantom{#1}} #5
\end{align*}%
}


%%%%%%%%%%%%%%%%%%%%%%%%%%%%%%%%%%%%%%%%%%%%%%
%               Ensembles.                   %
%%%%%%%%%%%%%%%%%%%%%%%%%%%%%%%%%%%%%%%%%%%%%%
\newcommand*{\NN}{\mathbf{N}}
\newcommand*{\ZZ}{\mathbf{Z}}
\newcommand*{\QQ}{\mathbf{Q}}
%%%%%%%%%%%%%%%%%%%%%%%%%%%%%
%   PAGE DE GARDE            %
%%%%%%%%%%%%%%%%%%%%%%%%%%%%%
\newcommand{\HRule}{\rule{\paperwidth}{0.5mm}} % trait, régler eppaisseur
\newcommand*{\theuniversity}{Université de Toulon}
\newcommand*{\theyearname}{Licence de Mathématiques, parcours mathématiques, 
2\ieme~année}
\newcommand*{\thesupervisor}{Joachim \bsc{Asch}}
\author{Anastasiia \bsc{Chernetcova}~\&~Tom \bsc{Domenge}}
\title{Projet TER 2020\par
            Dénombrabilité : \par $\NN,2\NN \text{ \bfseries\& } \QQ$\par}


%%%%%%%%%%%%%%%%%%%%%%%%%%%%%%%%%%%%%%%%%%%%%%
%   	Marges et note en marge.   			 %
%%%%%%%%%%%%%%%%%%%%%%%%%%%%%%%%%%%%%%%%%%%%%%
\setulmargins{*}{*}{*}
\setheaderspaces{*}{*}{*}
\setlrmargins{*}{*}{*}
\setheadfoot{\headheight}{\footskip}
\checkandfixthelayout[nearest]
\renewcommand*{\sideparfont}{\itshape\footnotesize}
\renewcommand*{\sideparform}{\ifmemtortm\RaggedRight\else\RaggedLeft\fi}

%%%%%%%%%%%%%%%%%%%%%%%%%%%%%%%%%%%%%%%%%%%%%%%%
%            Numérotation (classe memoir)       %
%%%%%%%%%%%%%%%%%%%%%%%%%%%%%%%%%%%%%%%%%%%%%%%%

\setsecnumdepth{subsubsection}
\renewcommand{\cftpartaftersnum}{.}
\renewcommand{\cftchapteraftersnum}{.}
\renewcommand{\cftpartdotsep}{\cftdotsep}
\renewcommand{\cftchapterdotsep}{\cftdotsep}% Chapters should use dots in ToC
\aliaspagestyle{title}{empty}
\aliaspagestyle{part}{empty}
\addto\captionsfrench{\renewcommand{\appendixpagename}{Annexes}}
\addto\captionsfrench{\renewcommand{\appendixtocname}{Annexes}}

%%%%%%%%%%%%%%%%%%%%%%%%%%%%%%%%%%%%%%%%%%%%%%%%
%            Polices                           %
%%%%%%%%%%%%%%%%%%%%%%%%%%%%%%%%%%%%%%%%%%%%%%%%
\setmainfont{TeX Gyre Pagella}
\setmathfont{Asana Math}
\setmathfont[range={"0000-"0001,"0020-"007E,
                    "00A0,"00A7-"00A8,"00AC,"00AF,"00B1,"00B4-"00B5,"00B7,
                    "00D7,"00F7,
                    "0131,
                    "0237,"02C6-"02C7,"02D8-"02DA,"02DC,
                    "0300-"030C,"030F,"0311,"0323-"0325,"032E-"0332,"0338,
                    "0391-"0393,"0395-"03A1,"03A3-"03A8,"03B1-"03BB,
                    "03BD-"03C1,"03C3-"03C9,"03D1,"03D5-"03D6,"03F5,
                    "2016,"2018-"2019,"2021,"2026-"202C,"2032-"2037,"2044,
                    "2057,"20D6-"20D7,"20DB-"20DD,"20E1,"20EE-"20EF,
                    "210B-"210C,"210E-"2113,"2118,"211B-"211C,"2126-"2128,
                    "212C-"212D,"2130-"2131,"2133,"2135,"2190-"2199,
                    "21A4,"21A6,"21A9-"21AA,"21BC-"21CC,"21D0-"21D5,
                    "2200,"2202-"2209,"220B-"220C,"220F-"2213,"2215-"221E,
                    "2223,"2225,"2227-"222E,"2234-"2235,"2237-"223D,
                    "2240-"224C,"2260-"2269,"226E-"2279,"2282-"228B,"228E,
                    "2291-"2292,"2295-"2299,"22A2-"22A5,"22C0-"22C5,
                    "22DC-"22DD,"22EF,"22F0-"22F1,
                    "2308-"230B,"2320-"2321,"2329-"232A,"239B-"23AE,
                    "23DC-"23DF,
                    "27E8-"27E9,"27F5-"27FE,"2A0C,"2B1A,
                    "1D400-"1D433,"1D49C,"1D49E-"1D49F,"1D4A2,"1D4A5-"1D4A6,
                    "1D4A9-"1D4AC,"1D4AE-"1D4B5,"1D4D0-"1D4E9,"1D504-"1D505,
                    "1D507-"1D50A,"1D50D-"1D514,"1D516-"1D51C,"1D51E-"1D537,
                    "1D56C-"1D59F,"1D6A8-"1D6B8,"1D6BA-"1D6D2,"1D6D4-"1D6DD,
                    "1D6DF,"1D6E1,"1D7CE-"1D7D7
                   }]{euler.otf}
\setmathfont[range=up/{greek,Greek}, script-features={}, sscript-features={}
            ]{euler.otf}
\setmathfont[range=up/{latin,Latin}, script-features={}, sscript-features={}
            ]{euler.otf}
\setmathfont[range={bfup/{latin, Latin, greek, Greek}, frak, bffrak},
             script-features={}, sscript-features={}
            ]{euler.otf}
\setmathfont[range={up/num, bfup/num, it, bfit,
                    sfup, sfit, bfsfup, bfsfit, tt}
                  ]{Asana Math}
\setmathfont[range={bb},Scale=MatchUppercase]{STIX Two Math}
\setmathfont[range={scr,cal},Scale=MatchUppercase]{TeX Gyre Pagella Math}
 \setmathfont[range=bfcal, Scale=MatchUppercase, Alternate]{Asana Math}
%%%%%%%%%%%%%%%%%%%%%%%%%%%%%%%%%%%%%%%%%%%%%%
%   Opérateurs (dans le style de max etc.   %
%%%%%%%%%%%%%%%%%%%%%%%%%%%%%%%%%%%%%%%%%%%%%%
\DeclareMathOperator{\Card}{Card} % s'utilise avec \Card en mode maths
\newcommand{\square}{\mdlgwhtsquare}

\usetikzlibrary{arrows.meta}
\graphicspath{{./figures/}}
\tikzset{|/.tip={Bar[width=.8ex,round]}}
\makeatletter
\newcommand{\diagdenomb}{\@ifstar{\@diagdenomb}{\@@diagdenomb}}
\newcommand{\@diagdenomb}{\begin{figure}[!htbp]
    \centering
\begin{tikzpicture}
\tikzstyle{keepstyle} =[rectangle, rounded corners, draw, fill=white]
\node at (0,0) {$\vdots$};
\node[keepstyle] (51) at (0,1) {$\frac{5}{1}$};
\node[keepstyle] (41) at (0,2) {$\frac{4}{1}$};
\node[keepstyle] (31) at (0,3) {$\frac{3}{1}$};
\node[keepstyle] (21) at (0,4) {$\frac{2}{1}$};
\node[keepstyle] (11) at (0,5) {$\frac{1}{1}$};
\node at (1,0) {$\vdots$};
\node[keepstyle] (52) at (1,1) {$\frac{5}{2}$};
\node at (1,2) {$\frac{4}{2}$};
\node[keepstyle] (32) at (1,3) {$\frac{3}{2}$};
\node at (1,4) {$\frac{2}{2}$};
\node[keepstyle] (12) at (1,5) {$\frac{1}{2}$};
\node at (2,0) {$\vdots$};
\node at (2,1) {$\frac{5}{3}$};
\node[keepstyle] (43) at (2,2) {$\frac{4}{3}$};
\node at (2,3) {$\frac{3}{3}$};
\node[keepstyle] (23) at (2,4) {$\frac{2}{3}$};
\node[keepstyle] (13) at (2,5) {$\frac{1}{3}$};
\node at (3,0) {$\vdots$};
\node at (3,1) {$\frac{5}{4}$};
\node at (3,2) {$\frac{4}{4}$};
\node[keepstyle] (34) at (3,3) {$\frac{3}{4}$};
\node at (3,4) {$\frac{2}{4}$};
\node[keepstyle] (14) at (3,5) {$\frac{1}{4}$};
\node at (4,0) {$\vdots$};
\node  at (4,1) {$\frac{5}{5}$};
\node at (4,2) {$\frac{4}{5}$};
\node at (4,3) {$\frac{3}{5}$};
\node[keepstyle] (25) at (4,4) {$\frac{2}{5}$};
\node[keepstyle] (15) at (4,5) {$\frac{1}{5}$};
\node at (5,0) {$\vdots$};
\node  at (5,1) {$\frac{5}{6}$};
\node at (5,2) {$\frac{4}{6}$};
\node at (5,3) {$\frac{3}{6}$};
\node at (5,4) {$\frac{2}{6}$};
\node[keepstyle] (16) at (5,5) {$\frac{1}{6}$};
\node at (6,1) {$\cdots$};
\node at (6,2) {$\cdots$};
\node at (6,3) {$\cdots$};
\node at (6,4) {$\cdots$};
\node at (6,5) {$\cdots$};
\draw [-latex,red, thick] (11) -- (12);
\draw [-latex, red, thick] (12) -- (21);
\draw [-latex, red, thick] (21) -- (31);
\draw [-latex, red, thick] (31) -- (13);
\draw [-latex, red, thick] (13) -- (14);
\draw [-latex, red, thick] (14) -- (23);
\draw [-latex, red, thick] (23) -- (32);
\draw [-latex, red, thick] (32) -- (41);    
\draw [-latex, red, thick] (41) -- (51);
\draw [-latex, red, thick] (51) -- (15);
\draw [-latex, red, thick] (15) -- (16);
\draw [-latex, red, thick] (16) -- (25);
\draw [-latex, red, thick] (25) -- (34);
\draw [-latex, red, thick] (34) -- (43);
\draw [-latex, red, thick] (43) -- (52);
\end{tikzpicture}
\end{figure}
}
\newcommand{\@@diagdenomb}{\begin{figure}[htbp]
    \centering
\begin{tikzpicture}
\tikzstyle{keepstyle} =[rectangle, rounded corners, draw, fill=white]
\node at (0,0) {$\vdots$};
\node[keepstyle] (51) at (0,1) {$\frac{5}{1}$};
\node[keepstyle] (41) at (0,2) {$\frac{4}{1}$};
\node[keepstyle] (31) at (0,3) {$\frac{3}{1}$};
\node[keepstyle] (21) at (0,4) {$\frac{2}{1}$};
\node[keepstyle] (11) at (0,5) {$\frac{1}{1}$};
\node at (1,0) {$\vdots$};
\node[keepstyle] (52) at (1,1) {$\frac{5}{2}$};
\node at (1,2) {$\frac{4}{2}$};
\node[keepstyle] (32) at (1,3) {$\frac{3}{2}$};
\node at (1,4) {$\frac{2}{2}$};
\node[keepstyle] (12) at (1,5) {$\frac{1}{2}$};
\node at (2,0) {$\vdots$};
\node at (2,1) {$\frac{5}{3}$};
\node[keepstyle] (43) at (2,2) {$\frac{4}{3}$};
\node at (2,3) {$\frac{3}{3}$};
\node[keepstyle] (23) at (2,4) {$\frac{2}{3}$};
\node[keepstyle] (13) at (2,5) {$\frac{1}{3}$};
\node at (3,0) {$\vdots$};
\node at (3,1) {$\frac{5}{4}$};
\node at (3,2) {$\frac{4}{4}$};
\node[keepstyle] (34) at (3,3) {$\frac{3}{4}$};
\node at (3,4) {$\frac{2}{4}$};
\node[keepstyle] (14) at (3,5) {$\frac{1}{4}$};
\node at (4,0) {$\vdots$};
\node  at (4,1) {$\frac{5}{5}$};
\node at (4,2) {$\frac{4}{5}$};
\node at (4,3) {$\frac{3}{5}$};
\node[keepstyle] (25) at (4,4) {$\frac{2}{5}$};
\node[keepstyle] (15) at (4,5) {$\frac{1}{5}$};
\node at (5,0) {$\vdots$};
\node  at (5,1) {$\frac{5}{6}$};
\node at (5,2) {$\frac{4}{6}$};
\node at (5,3) {$\frac{3}{6}$};
\node at (5,4) {$\frac{2}{6}$};
\node[keepstyle] (16) at (5,5) {$\frac{1}{6}$};
\node at (6,1) {$\cdots$};
\node at (6,2) {$\cdots$};
\node at (6,3) {$\cdots$};
\node at (6,4) {$\cdots$};
\node at (6,5) {$\cdots$};
\draw [-latex,red, thick] (11) -- (12);
\draw [-latex, red, thick] (12) -- (21);
\draw [-latex, red, thick] (21) -- (31);
\draw [-latex, red, thick] (31) -- (13);
\draw [-latex, red, thick] (13) -- (14);
\draw [-latex, red, thick] (14) -- (23);
\draw [-latex, red, thick] (23) -- (32);
\draw [-latex, red, thick] (32) -- (41);    
\draw [-latex, red, thick] (41) -- (51);
\draw [-latex, red, thick] (51) -- (15);
\draw [-latex, red, thick] (15) -- (16);
\draw [-latex, red, thick] (16) -- (25);
\draw [-latex, red, thick] (25) -- (34);
\draw [-latex, red, thick] (34) -- (43);
\draw [-latex, red, thick] (43) -- (52);
\end{tikzpicture}
\caption{Dénombrabilité de $\mathbb{Q}$\protect\cite{Tikz-Count}}
    \label{fig:denombQ}
\end{figure}
}
\tikzset{    >=stealth,
    bullet/.style={
        fill=black,
        circle,
        minimum width=1pt,
        inner sep=1pt
    },
    projection/.style={
        thick,
        shorten <=2pt,
        shorten >=2pt
    },
    every fit/.style={
        ellipse,
        draw,
        inner sep=2pt
        }}
\makeatother
\graphicspath{{./figures/}} % pour les images, plus besoin d'écrire "./figures"
\newcommand{\diagfonctbij}{
	\begin{figure}[htbp]
	\centering
    \begin{tikzpicture}
	\node[bullet,fill=bleu,label=left:$a$] (a1) at (0,4) {};    
	\node[bullet,fill=bleu,label=left:$b$] (a2) at (0,3) {};    
	\node[bullet,fill=bleu,label=left:$c$] (a3) at (0,2) {};
	\node[bullet,fill=bleu,label=left:$d$] (a4) at (0,1) {};%
	%
	\node[bullet,fill=roug,label=right:$1$] (b1) at (4,4) {};
	\node[bullet,fill=roug,label=right:$2$] (b2) at (4,3) {};
	\node[bullet,fill=roug,label=right:$3$] (b3) at (4,2) {};
	\node[bullet,fill=roug,label=right:$4$] (b4) at (4,1) {};%
%
	\node[fill=bleu,draw=bleu,fill opacity=0.3,fit= (a1) (a2) (a3) (a4),minimum width=2cm, label=above:$E$] (X) {} ;
	\node[fill=roug,draw=roug,fill opacity=0.3,fit= (b1) (b2) (b3) (b4),minimum width=2cm, label=above:$F$] (Y) {} ;  %
%
	\draw[|->,projection] (a1) to[out=20, in=150] (b4);
	\draw[|->,projection] (a2) to[out=20, in=160] (b2);
	\draw[|->,projection] (a3) to[out=20, in=170] (b1);
	\draw[|->,projection] (a4) to[out=20, in=150] (b3);
%
	\draw[->,thick,shorten <=1cm,shorten >=1cm] (X.north) -- node [midway,above,align=center]{$f$} (Y.north);
	\end{tikzpicture}%
%
	\caption{diagramme sagittal d'une application bijective.}
	\label{fig:fonctbij}
\end{figure}
}
\newcommand{\diagfonctinj}{
	\begin{figure}[htbp]
	\centering
    \begin{tikzpicture}
	\node[bullet,fill=bleu,label=left:$a$] (a1) at (0,4) {};    
	\node[bullet,fill=bleu,label=left:$b$] (a2) at (0,3) {};    
	\node[bullet,fill=bleu,label=left:$c$] (a3) at (0,2) {};%
	%
	\node[bullet,fill=roug,label=right:$1$] (b1) at (4,4) {};
	\node[bullet,fill=roug,label=right:$2$] (b2) at (4,3) {};
	\node[bullet,fill=roug,label=right:$3$] (b3) at (4,2) {};
	\node[bullet,fill=roug,label=right:$4$] (b4) at (4,1) {};%
%
	\node[fill=bleu,draw=bleu,fill opacity=0.3,fit= (a1) (a2) (a3),minimum width=2cm, label=above:$X$] (X) {} ;
	\node[fill=roug,draw=roug,fill opacity=0.3,fit= (b1) (b2) (b3) (b4),minimum width=2cm, label=above:$Y$] (Y) {} ;  %
%
	\draw[|->,projection] (a1) to[out=20, in=150] (b4);
	\draw[|->,projection] (a2) to[out=20, in=160] (b2);
	\draw[|->,projection] (a3) to[out=20, in=170] (b1);
%
	\draw[->,thick,shorten <=1cm,shorten >=1cm] (X.north) -- node [midway,above,align=center]{$f$} (Y.north);
	\end{tikzpicture}%
%
	\caption{diagramme sagittal d'une application injective.}
	\label{fig:fonctinj}
\end{figure}
}
\newcommand{\diagfonctsurj}{
	\begin{figure}[htbp]
	\centering
    \begin{tikzpicture}
	\node[bullet,fill=bleu,label=left:$a$] (a1) at (0,4) {};    
	\node[bullet,fill=bleu,label=left:$b$] (a2) at (0,3) {};    
	\node[bullet,fill=bleu,label=left:$c$] (a3) at (0,2) {};%
	\node[bullet,fill=bleu,label=left:$c$] (a4) at (0,1) {};%
	%
	\node[bullet,fill=roug,label=right:$1$] (b1) at (4,4) {};
	\node[bullet,fill=roug,label=right:$2$] (b2) at (4,3) {};
	\node[bullet,fill=roug,label=right:$3$] (b3) at (4,2) {};
%
	\node[fill=bleu,draw=bleu,fill opacity=0.3,fit= (a1) (a2) (a3) (a4),minimum width=2cm, label=above:$X$] (X) {} ;
	\node[fill=roug,draw=roug,fill opacity=0.3,fit= (b1) (b2) (b3),minimum width=2cm, label=above:$Y$] (Y) {} ;  %
%
	\draw[|->,projection] (a1) to[out=20, in=150] (b3);
	\draw[|->,projection] (a2) to[out=20, in=160] (b2);
	\draw[|->,projection] (a3) to[out=20, in=170] (b1);
	\draw[|->,projection] (a4) to[out=20, in=170] (b1);
%
	\draw[->,thick,shorten <=1cm,shorten >=1cm] (X.north) -- node [midway,above,align=center]{$f$} (Y.north);
	\end{tikzpicture}%
%
	\caption{diagramme sagittal d'une application surjective.\\ On a $|X|\geq|Y|$.}
	\label{fig:fonctsurj}
\end{figure}
}
\newcommand{\diagcompfonct}{
	\begin{figure}[htbp]
		\centering
		\begin{tikzpicture}
			\node[bullet,label=left:$a$] (a1) at (0,4) {};    
			\node[coordinate,label] (a2) at (0,3) {};    
			\node[coordinate,label] (a3) at (0,2) {};
			\node[coordinate,label] (a4) at (0,1) {};
			\node[coordinate,fill=roug] (b1) at (4,4) {};
			\node[coordinate,fill=roug] (b2) at (4,3) {};
			\node[coordinate,fill=roug] (b3) at (4,2) {};
			\node[bullet,fill=roug,label=right:$4$] (b4) at (4,1) {};
			\node[coordinate,fill=vert] (c1) at (8,4) {};
			\node[bullet,fill=vert,label=right:pair] (c2) at (8,3) {};
			\node[coordinate,fill=vert] (c3) at (8,2) {};
			\node[coordinate,fill=vert] (c4) at (8,1) {};
			\node[fill=bleu,draw=bleu,fill opacity=0.3,fit= (a1) (a2) (a3) (a4),minimum width=2cm, label=above:$X$] (X) {} ;
			\node[fill=roug,draw=roug,fill opacity=0.3,fit= (b1) (b2) (b3) (b4),minimum width=2cm, label=above:$Y$] (Y) {} ;  
			\node[fill=vert,draw=vert,fill opacity=0.3,fit= (c1) (c2) (c3) (c4),minimum width=2cm, label=above:$Z$] (Z) {} ;  
			\draw[|->,projection] (a1) to[out=-10, in=150] (b4);
			%
			\draw[|->,projection] (b4) to[out=50, in=180] (c2);
			\draw[|->,projection,dashed] (a1) to[out=-25, in=180] (c2);
			%
			\draw[->,thick,shorten <=1cm,shorten >=1cm] (X.north) -- node [midway,above,align=center]{$f$} (Y.north);
			\draw[->,thick,shorten <=1cm,shorten >=1cm] (Y.north) -- node [midway,above,align=center]{$g$} (Z.north);
			\draw[->,thick,dashed,shorten <=.5cm,shorten >=.5cm] (X.north) to[out=50, in=130] node [midway,above=.25em,align=center]{$g\circ f$} (Z.north);
		\end{tikzpicture}
		\caption{Composition d'applications, détail pour un antécédent.}
		\label{fig:diagcompfonct}
	\end{figure}
}
\begin{document}
\begin{titlingpage}
\newgeometry{left=2.1cm,right=2.1cm,top=2.1cm,bottom=2.1cm}
\vspace*{\fill}%
\begin{center}
\vspace*{\fill}%
\makebox[\linewidth]{\HRule} % demande à latex l'espace pour le trait
\parbox[t]{\textwidth}{\addvspace{\parskip} % on ajoute ce qu'il faut d'espace pour sous le trait
\centering\huge\bfseries\thetitle}
 \null%Espace obligatoire pour LaTeX (boite vide) sinon l'espace est ignoré
\vspace*{\parskip}
\makebox[\linewidth]{m\HRule}
\vspace*{\fill}
\diagdenomb*
\vspace{\fill}
\par \Large Rédigé par \theauthor\par Supervisé par \thesupervisor \par
\vspace*{\fill}
\large{\theyearname.\par \theuniversity, Version du \thedate}
\end{center}
\restoregeometry
\end{titlingpage}

\frontmatter
\tableofcontents
\part{Introduction}
\chapter{Avant-propos}

La notion de l'infini (noté $\infty$) apparaît pour la première fois lorsqu'on étudie l'ensemble des entiers naturels. C'est une définition intuitive de l'infini. L'infini n'est pas un nombre, il n'obéit pas aux lois usuelles $+$ et $\times$ de la même manière que les nombres réels. En mathématiques, l'une des manières les plus simples de caractériser l'infini est celle de la théorie des ensembles. L'une des propriétés principales d'un ensemble est sa taille. Comme nous allons le voir, les ensembles mathématiques peuvent être finis ou infinis : L'étude des ensembles infinis est révélatrice de paradoxes, l'un des plus célèbres  énonce qu'une partie stricte d'un ensemble infini peut contenir autant d'éléments que l'ensemble lui-même. C'est le paradoxe de l'hôtel de \bsc{Hilbert} (1924).

On peut alors se poser des questions fondamentales à l'étude de l'infini~: comment peut on caractériser les ensembles infinis ? Est-ce qu'il y a un infini << plus grand >> ou << plus petit >> que celui des nombres ?  

Dans la première partie, nous allons rappeler ce que sont les ensembles, cruciaux pour l'étude de l'infini du point de vue ensembliste.
Dans la seconde partie, nous nous consacrerons à l'étude de dénombrabilité d'ensembles usuels et aux propriétés d'ensembles dénombrables.

Notre projet devait initialement s'articuler autour de l'ouvrage~\cite{livre_ter}. Nous avons choisi une approche différente : nous voulions insister davantage sur l'aspect formel de la dénombrabilité, c'est pourquoi nous avons introduit des définitions, des notations, des propositions démontrées pour souligner notre propos. \`A notre avis, ce travail est nécessaire pour bien aborder la notion de la dénombrabilité et pour ne pas se perdre dans les différentes notions. Ainsi, nous avons choisi d'en donner une définition rigoureuse. Bien que cette notion soit intuitive, il s'avère difficile de l'expliquer rigoureusement avec des mots du langage << courant >> (non mathématique). En ce sens, le langage mathématique offre une exactitude que la langue naturelle n'a pas.

\epigraph{%
	\foreignlanguage{german}{\enquote{\itshape Der Mathematiker abstrahiert gänzlich von der Beschaffenheit der Gegenstände und dem Inhalt ihrer Relationen; er hat es bloß mit der Abzählung und Vergleichung der Relationen unter sich zu tun}}\newline \newline
\enquote{Le mathématicien fait abstraction de la nature des objets qu'il manipule, et des relations qu'ils entretiennent ; il lui suffit de les passer en revue et de les comparer entre elles}
}{Carl Friedrich \bsc{Gauss}\footnotemark~(1831)}\footnotetext{Voir~\cite{gauss_cite}}
\mainmatter
\part{Préliminaires au dénombrement~: Comptage, ensembles, bijection}
\chapter{Rappels de théorie des ensembles \& notion de fonction}
Notion de fonction, injections, surjections, bijection. 
Lien même nombre d'objets $ \iff$ bijection ensembliste. 
\clearpage
\diagfonctinj
\diagfonctsurj
\diagfonctbij
\diagcompfonct

\chapter{\texorpdfstring{Définition de $\mathbb{N}$}{Définition de N}}
Axiomatique de \bsc{Peano}, Récurrence

\chapter{Définition de la cardinalité}
\chapter{Des parties infinies ? } 
L'intuition nous pousse à croire que, comme dans le cas fini, toute partie de $\mathbb{N}$, stricte, est finie. Il n'en est rien, comme Euclide l'a démontré, vers~-300~av.~J.C. 

On se propose donc ici de reformuler sa démonstration en terme modernes. 
La démonstration du théorème suivant est rejetée en \cref{annexe:thmfondarith}.
\begin{defb}{Divisibilité}{div}
Soit $a,b$ deux entiers naturels. On dit que $a$ \emph{divise} $b$ ou que $b$ est un \emph{multiple} de $a$, et on note $a\mid b$, s'il existe un entier naturel $k$ tel que : $$b=ka \xLeftrightarrow{ a \mathrel{\neq} 0 } \frac{b}{a}=k.$$
\end{defb}
\begin{remarkb}{}{divinf}
0 ne divise que 0. De plus, si $a$ divise $b$, alors $a\leq b$. Par conséquent, 1 est le seul entier naturel qui divise 1.
\end{remarkb}
\begin{defb}{Nombre premiers entre eux}{coprem}
On dit que deux nombres entiers $n,p$ sont \emph{premiers entre eux} ou \emph{copremiers} s'ils n'ont pas de diviseur commun, mis à part 1. On dit alors que $n$ \emph{est premier} à ou avec $q$.
\end{defb}
\begin{defb}{Nombre premier}{prem}
On dit qu'un nombre est premier, s'il n'a pas de diviseurs autre que 1 et lui-même. Dans le cas contraire, on le dit \emph{composé}.

Par convention\footnotemark, 1 n'est ni premier ni composé.
\end{defb}
 \footnotetext{Cette convention est récente dans l'histoire des mathématiques : 1 était considéré premier jusqu'à la fin du XIX\ieme~siècle, voire au début du XX\ieme.}
La remarque suivante justifie l'emploi du même vocabulaire : 
\begin{remarkb}{}{prem-equiv}
Un nombre est premier si et seulement s'il est copremier avec tous ceux qui le précèdent, c'est-à-dire s'il n'a aucun diviseur commun avec un entier qui lui est inférieur. 
\end{remarkb}
\begin{lemmab}{}{divprod}
Si $n$ divise deux entiers $a,b$, il divise aussi leur produit $ab$. 
\end{lemmab}
\begin{proof}
Comme $n$ divise $a$, il existe un entier $k$ tel que $a=kn$, de même, comme $n$ divise $b$, il existe un entier $k'$ tel que $b=k'n$. Ainsi, \[ab=\underbrace{knk'}_{\eqdef k_2\in \mathbb{N}}\times n.\]
\end{proof}
Le lemme suivant est \textbf{fondamental} dans ce qui suit.
\begin{lemmab}{Théorème fondamental de l'arithmétique}{Fonda-Arith}
  Tout entier naturel non-nul peut être écrit comme un produit de nombres premiers d'une unique façon, à l'ordre près des facteurs.
\end{lemmab}

La démonstration de ce théorème est rejetée en~\cref{annexe:thmfondarith}.

Ces rappels balayés, on peut démontrer le résultat d'Euclide 

\begin{theoremb}{Infinité des nombres premiers}{nombpreminf} 
Toute liste finie de nombre premiers est nécessairement incomplète.

Autrement dit, il existe une infinité de nombres premiers.
\end{theoremb}
\begin{proof}
\reversemarginpar
On démontre le résultat par récurrence. Une liste vide (à 0 éléments) ne contient aucun nombre premier, elle est donc incomplète car 2 est premier. 
Supposons que l'on dispose d'une liste finie de $n$ nombres premiers, pour $n$ un entier naturel quelconque. En triant notre liste dans l'ordre croissant, on obtient une suite de $n$ entiers, que l'on note $(p_i)_{i\in \left\lBrack1;n\right\rBrack}$. Alors,on peut considérer, l'entier $N$ défini par : \[ N=\prod_{i=1}^{n} p_i\]\sidepar{\vspace*{-3.5\baselineskip}\itshape \footnotesize Donc $N$ est égal à Pi de 1 à $n$ de $p_i$}
Si notre suite est vide, on pose $N=2$.
Les termes $(p_i)_{i\in \left\lBrack1;n\right\rBrack}$ sont les facteurs premiers de $N$, par le \cref{lem:Fonda-Arith}.
\sidepar{\itshape \footnotesize $2$ est le plus petit nombre premier.}
Dans les deux cas, d'après le \cref{lem:ncopremsn}, $N+1$ est premier avec $N$. Comme $N\geq 2$,  alors $N+1\geq 3$, et $N$ admet donc un facteur premier, en vertu du~\cref{lem:Fonda-Arith}. Mais alors, d'après la \cref{rem:prem-equiv}, un des facteurs premiers de $N+1$ ne figure dans notre liste. 

Par récurrence, le résultat est démontré.
\end{proof}
\part{\texorpdfstring{Dénombrabilité de 2$\mathbb{N}$, $\mathbb{Z}$, $\mathbb{Q}$}{Dénombrabilité de Q}} \chapter{Définition et propriétés de la dénombrabilité}
\section{Ensembles finis et infinis} \sidepar{$\to$  def cardinalité ?}

\begin{defb}{Finitude}{fini}
     Soit $E$ un ensemble. 
	    
	On dit que $E$ est fini si $E = \emptyset$ ou s'il existe une bijection $\varphi~: \{1,\dots,n \} \to E$, avec $n \in \mathbb{N}$. 
		
	Dans ce cas, on dit que $E$ est de cardinal $n$ (ou de puissance $n$), et on note le cardinal $\Card(E)$. 
\end{defb}

\begin{defb}{Equipotence}{equip}
    On dit que $E$ et $F$ sont équipotents ssi $\Card(E) = \Card(F)$ s' il existe $\varphi~: E \to F$ bijective. Dans ce cas, on note $E \mathrel{\approx} F$. 
\end{defb}

\begin{theoremb}{}{mmpuiss}
    $ \mathrel{\approx}$ est une relation d'équivalence. 
\end{theoremb}

\begin{proof}
    Soient $E, F, G$, trois ensembles.
	\begin{enumerate} 
		\item $E \mathrel{\approx} E$, car l'application identité est une bijection. $\mathrel{\approx}$ est réflexive.
		\item $E \mathrel{\approx} F$ signifie qu'il existe une bijection de E dans F. Alors, l'inverse de cette bijection est une application de F dans E, elle-même bijective.
		\item Si $E \mathrel{\approx} F$ et $F \mathrel{\approx} G$ alors il existe une bijection de E dans F, $\phi$, et une bijection de F dans g,$\psi$. Par composition, $\psi\circ\phi$ est une bijection de E dans G, ie $E \mathrel{\approx} G$. $\mathrel{\approx}$ est transitive. 
	\end{enumerate}
	$\mathrel{\approx}$ est bien une relation d'équivalence.
\end{proof}

\begin{theoremb}{}{assequiv}
	Soient $E, F$ deux ensembles \textbf{finis} et $\varphi~: E \to F$ une application. Si $E \mathrel{\approx} F$ alors les 3 assertions suivantes sont équivalentes~:
	
	\begin{enumerate}
		\item $\varphi$ est injective ;
		\item $\varphi$ est surjective ;
		\item $\varphi$ est bijective. 
	\end{enumerate}
\end{theoremb}

\begin{defb}{Ensemble infini}{ifini}
	Soit $E$ un ensemble.
	
	On dit que $E$ est infini ssi $E$ n'est pas fini.
\end{defb}

\begin{theoremb}{}{partn} 
	Toute partie de $\mathbb{N}$ est finie ssi elle est majorée. 
\end{theoremb}



\begin{theoremb}{}{ninfini}
	$\mathbb{N}$ est infini. 
\end{theoremb}


\begin{proof}
	Si $\mathbb{N}$ était fini, d'après le \cref{thm:partn}, $\mathbb{N}$ admettrait un majorant. Autrement dit \[\exists k \in \mathbb{N}, \forall x \in \mathbb{N}, x \leq k.\]

	Or la construction de $\mathbb{N}$ implique que si $ k \in \mathbb{N}$, alors $k + 1 \in \mathbb{N}$ et $\exists x \in \mathbb{N}, x > k, x= k+1$, d'où une contradiction. 
	
	Donc $\mathbb{N}$ est infini.
\end{proof}

\section{Ensembles dénombrables}

\begin{defb}{Ensemble démombrable, au plus dénombrable}{denomb}

	 Soit $E$ un ensemble infini. 
	 
	 On dit que $E$ est dénombrable ssi il existe une application $\varphi~: \mathbb{N} \to E $ bijective. 
	 
	 Un ensemble fini ou dénombrable est dit au plus dénombrable. 
\end{defb}

On note par la suite $\aleph_0$,lire \emph{alèf zéro}\footnote{Aleph, première lettre  de l'alphabet hébreu}, le cardinal de $\mathbb{N}$. 
\clearpage
\begin{samepage}
\section{Propriétés usuelles de dénombrabilité}

\begin{theoremb}{}{parties}

	Soient $E$ et $F$ deux ensembles. 
	Si $E \subset F$ et si $F$ est dénombrable, alors $E$ est au plus dénombrable.
\end{theoremb}

\begin{theoremb}{Union de deux ensembles dénombrables}{union}
    Soit $(A_i)_{i \in I}$ une famille dénombrable d'ensemble dénombrable. 
	
	Alors $A= \bigcup_{i \in I} A_i$ est dénombrable. 
\end{theoremb}


\begin{theoremb}{Produit cartésien d'ensembles dénombrables}{cartesien}
	Soient $A, B, A_1, \dots, A_i, \dots, A_n$ des ensembles dénombrables. 
	
	\begin{enumerate}
		\item Le produit cartésien $A \times B$ est dénombrable. 
		\item Plus généralement, tout produit \textbf{fini} $\prod_{i=1}^{n} A_i $ est dénombrable.
	\end{enumerate}
\end{theoremb}
\begin{proof}
	Les  \cref{thm:union,thm:cartesien} seront démontrées par la suite dans la \cref{sec:denombrabilite_usuelle}~.
	
	% saut de ligne pour virer le cqfd
\end{proof}
\end{samepage}

\section{\texorpdfstring{Dénombrabilité de 2$\mathbb{N}$}{Dénombrabilité de 2N}}
\label{sec:denombrabilite_usuelle}
Naturellement on pourrait penser que les ensembles inclus dans $\mathbb{N}$ ont une cardinal inférieur à $\aleph_0$. Contre toute attente, voici un résultat sur la dénombrabilité de  $2\mathbb{N}$, l'ensemble des entiers naturels pairs.

\begin{theoremb}{Dénombrabilité de 2$\mathbb{N}$}{}
	L'ensemble 2$\mathbb{N}$ des entiers naturels pairs est dénombrable. 
\end{theoremb}

\begin{proof}
	On construit l'application \[ \begin{array}{cccc}
	\ & \mathbb{N}& \to& 2\mathbb{N} \\
	\varphi~: & n & \mapsto & 2n
	\end{array}.\]
	
	Cette application $\varphi $ est bijective car sa réciproque est $ \varphi^{-1}~: \begin{array}{ccc}
	2\mathbb{N} & \to & \mathbb{N}\\
	k & \mapsto & \frac{k}{2}
	\end{array}$. 
	
	2$\mathbb{N}$ est en bijection avec $\mathbb{N}$. 2$\mathbb{N}$ est bien un ensemble dénombrable de cardinal $\aleph_0$. 
\end{proof}

On peut introduire un résultat plus général qui couvre toute partie de $\mathbb{N}$. 

\begin{theoremb}{Dénombrabilité des parties infinies de $\mathbb{N}$}{}
	Toute partie $E \subset \mathbb{N}$ infinie a le même cardinal que $\mathbb{N}$. 
\end{theoremb}

\begin{proof}
	On construit, par récurrence, une bijection $\phi~: \mathbb{N} \to E$ telle que: 
	
	\[\left\lbrace \begin{array}{c}
	\phi(0) = \min \{x \in E\} \\
	\forall n \in \mathbb{N}, \phi(n+1) = \min \{x \in E, x > \phi(n) \}
	\end{array} \right.\]
	
	C'est bien une bijection~: elle est injective et surjective car $\forall k \in \mathbb{N}, k \leq \phi(k)$. 
\end{proof}

\section{\texorpdfstring{Dénombrabilité de $\mathbb{N}^2$}{Dénombrabilité de N²}}

Pour un ensemble $E$ fini de cardinal $n$, à moins que $\Card(E) = 1$, $\Card(E^2) \neq \Card(E)$. On pourrait penser que $\mathbb{N}^2 $ contient plus d'élements que $\mathbb{N}$. 

\begin{theoremb}{Dénombrabilité de $\mathbb{N}^2$}{n2_denombrable}
	$\mathbb{N}^2$ est dénombrable. 
\end{theoremb}

\begin{proof} 
	On peut compter les couples $(n,m) \in \mathbb{N}^2$ comme suit~: \par On représente ces couples sur un quart de plan cartésien. 
	On part du couple $(0, 0)$. Dans la ligne suivante, on range $(1,0) $ et $(0,1)$. On continue avec $(2,0), (1,1), (0,2)$ (on parcourt la diagonale du point de coordonnées $(n,0)$ pour arriver au point de coordonnées $(0,n)$). Ce procédé est illustré dans la  \cref{fig:n_croix_n}.
	\begin{figure}[htb]
		\centering
		\includegraphics[scale=0.3]{n_croix_n.png}
		\caption{Dénombrabilité de $\mathbb{N}^2$}
		\label{fig:n_croix_n}
	\end{figure}
	
	On obtient la séquence suivante~:
	
	\[\begin{array}{l}
		(0,0) \\
		(1,0), (0,1) \\
		(0,2), (1,1), (2,0) \\
		(3,0), (1,2), (2,1), (0,3) \\
		\cdots
	\end{array}.\]
	Plus formellement, l'application qui range les couples de cette manière est 
	\[\begin{array}{cccc}
		\ & \mathbb{N} \times \mathbb{N} & \to & \mathbb{N} \\
		\varphi~: & (n,m) & \mapsto & {\underbrace{\frac{(n+m)(n+m+1)}{2}}_{\text{la place du couple dans la ligne}}} +  {\underbrace{\vphantom{\frac{(n+m)(n+m+1)}{2}} m}_{\text{la place du couple dans la colonne}}}
	\end{array}.\]
	
	
	$\varphi$ est bien bijective. $\mathbb{N}^2$ est bien dénombrable.
\end{proof}


Par récurrence sur $k \in \mathbb{N}$, on peut montrer le théorème suivant :

\begin{theoremb}{}{nk_denombrable}
    $\mathbb{N}^k$ est dénombrable.
\end{theoremb}

\begin{proof}

Pour $k = 3$, on peut, en utilisant les notations du \cref{thm:n2_denombrable}, établir une bijection entre $\mathbb{N}^2$ et $\mathbb{N}^3$. Posons $h_3 : \mathbb{N}^2 \to \mathbb{N}^3$ telle que $h_3(n,m) = (n,\varphi^{-1}(m))$. $h_3$ est bien une bijection. 

Soit $k \in \mathbb{N}$ fixé. Supposons l'hypothèse vraie au rang $k$, i. e. $\mathbb{N}^k$ est dénombrable.

On construit une bijection entre $\mathbb{N}^2$ et $\mathbb{N}^{k+1}$. On pose $h_{k+1}(n_1, \dots, n_{k+1}) = (h_k(n_1, \dots, n_k),n_{k+1})$ où $h_k(n_1, \dots, n_k) \in \mathbb{N}$ car d'après l'hypothèse de récurrence, $\mathbb{N}^k$ est dénombrable. On a ainsi établi une bijection entre $\mathbb{N}^{k+1}$ et $\mathbb{N}^2$. Or $\mathbb{N}^2$ est dénombrable. Donc $\mathbb{N}^k$ est dénombrable. 

Par récurrence, $\forall k \in \mathbb{N}, \mathbb{N}^k$ est dénombrable. 

\end{proof}


De la dénombrabilité de $\mathbb{N}^2$, on peut démontrer le \cref{thm:cartesien}. 

\begin{proof}
	Si $E$ et $F$ sont dénombrables, il existe $\varphi~: E \to \mathbb{N}$ et $\psi~: F \to \mathbb{N}$ bijectives. On peut alors définir \[ \begin{array}{cccc}
	\ & A \times B & \to & \mathbb{N}^2 \\
	f~: & (x,y) & \mapsto & (\varphi(x), \psi(y))
	\end{array}.\]
	
\end{proof}

$f $ est bien bijective. Donc $A \times B$ est bien dénombrable. On procède par récurrence immédiate pour $k$ ensembles. 

Soit $\prod_{i =1}^k A_i$ le produit cartésien de $k$ ensembles dénombrables. Si $\forall i \in \{1,\dots,k\}, \varphi_i : A_i \to \mathbb{N}$ est une bijection, alors on peut construire l'application \[ \begin{array}{cccc}
	\ & \prod_{i=1}^k A_i & \to & \mathbb{N}^k\\
	f~: & (x_1,\dots,x_k) & \mapsto & (\varphi_1(x_1),\dots, \varphi_k(x_k))
	\end{array}.\] 
C'est une bijection de $\prod_{i=1}^k A_i $ vers $\mathbb{N}^k$ qui est dénombrable. Donc $\prod_{i=1}^kA_i$ est dénombrable.


On peut maintenant prouver le \cref{thm:union}. 


\begin{proof}
	En effet, on peut écrire $\forall i \in I, A_i = \{x_{n,i}, n \in \mathbb{N}\}$. Chaque élement $x_{n, i}$ est indéxé par sa position $n$ dans $A_i$ et la position $i$ de $(A_i)$ dans $I$. 
	
	Ainsi $A = \bigcup_{i \in I} A_i = \{x_{n, i}, (n,i) \in \mathbb{N} \times I\}$. Les élements $x_{n,i}$ sont indéxés par $\mathbb{N} \times I$ qui est un produit cartésien d'ensembles dénombrables ($I$ est une famille indéxée par $\mathbb{N}$, elle est bien dénombrable). Donc $A$ est dénombrable.
\end{proof}
\clearpage
\section{\texorpdfstring{Dénombrabilité de $\mathbb{Z}$ et de $\mathbb{Q}$}{}}

Intuitivement on pourrait penser que l'ensemble des entiers $\mathbb{Z}$ a un cardinal supérieur à $\mathbb{N}$ étant donné qu'il contient ``deux fois plus" d'élements~: les entiers naturels et les entiers négatifs. 

\begin{theoremb}{Dénombrabilité de $\mathbb{Z}$}{denz}
	L'ensemble $\mathbb{Z}$ des entiers est dénombrable. 
\end{theoremb}

\begin{proof}
L'idée de la démonstration revient à écrire l'ensemble des entiers, dans l'ordre suivant~:  $\mathbb{Z}=\left\lbrace 0,1,(-1),2,(-2),3,(-3) \dots\right\rbrace$ 

	Pour montrer que le cardinal de $\mathbb{Z}$ est bien $\aleph_0$, on définit donc l'application $\phi~: \mathbb{N} \to \mathbb{Z}$ telle que \[ \forall k \in \mathbb{N}, \
	\begin{dcases}
	\phi(2k)& = -k \\
	\phi(2k + 1)& = k+1
	\end{dcases}
\]
	
	C'est bien une bijection. Donc $\mathbb{Z}$ est dénombrable. 
\end{proof}
\begin{samepage}

\begin{theoremb}{Dénombrabilité de $\mathbb{Q}$}{denq}
	$\mathbb{Q}$ est dénombrable.
\end{theoremb}
\begin{proof}
	Tout $x \in \mathbb{Q}$ a un unique représentant irréductible $\frac{p}{q}$ où $p \wedge q = 1$. 
	
	Pour établir une bijection entre $\mathbb{Q}$ et $\mathbb{N}$, on commence par écrire un tableau où on range $\frac{p}{q}$ à la colonne $q$ et à la ligne $p$.  
	
	On parcourt le tableau comme c'est indiqué par les flèches dans la \cref{fig:denombQ}. 
	
	On part de $\frac{1}{1}$ dans le tableau. On se déplace à la colonne suivante pour atteindre $\frac{1}{2}$. On parcourt le tableau diagonalement vers le bas pour atteindre $\frac{2}{1}$, on descend d'une case pour obtenir $\frac{3}{1}$ et on parcourt le tableau diagonalement vers le haut jusqu'à atteindre $\frac{1}{3}$, etc. 
	
	On s'aperçoit que chaque nombre rationnel dans le tableau a un et un seul rang $r$ (sa place dans le tableau). L'application $\varphi$ qui range les rationnels $x \in \mathbb{Q}$ dans ce tableau est donc une bijection.
\end{proof}
\begin{figure}[!htbp]
    \centering
\begin{tikzpicture}
\tikzstyle{keepstyle} =[rectangle, rounded corners, draw, fill=white]
\node at (0,0) {$\vdots$};
\node[keepstyle] (51) at (0,1) {$\frac{5}{1}$};
\node[keepstyle] (41) at (0,2) {$\frac{4}{1}$};
\node[keepstyle] (31) at (0,3) {$\frac{3}{1}$};
\node[keepstyle] (21) at (0,4) {$\frac{2}{1}$};
\node[keepstyle] (11) at (0,5) {$\frac{1}{1}$};
\node at (1,0) {$\vdots$};
\node[keepstyle] (52) at (1,1) {$\frac{5}{2}$};
\node at (1,2) {$\frac{4}{2}$};
\node[keepstyle] (32) at (1,3) {$\frac{3}{2}$};
\node at (1,4) {$\frac{2}{2}$};
\node[keepstyle] (12) at (1,5) {$\frac{1}{2}$};
\node at (2,0) {$\vdots$};
\node at (2,1) {$\frac{5}{3}$};
\node[keepstyle] (43) at (2,2) {$\frac{4}{3}$};
\node at (2,3) {$\frac{3}{3}$};
\node[keepstyle] (23) at (2,4) {$\frac{2}{3}$};
\node[keepstyle] (13) at (2,5) {$\frac{1}{3}$};
\node at (3,0) {$\vdots$};
\node at (3,1) {$\frac{5}{4}$};
\node at (3,2) {$\frac{4}{4}$};
\node[keepstyle] (34) at (3,3) {$\frac{3}{4}$};
\node at (3,4) {$\frac{2}{4}$};
\node[keepstyle] (14) at (3,5) {$\frac{1}{4}$};
\node at (4,0) {$\vdots$};
\node  at (4,1) {$\frac{5}{5}$};
\node at (4,2) {$\frac{4}{5}$};
\node at (4,3) {$\frac{3}{5}$};
\node[keepstyle] (25) at (4,4) {$\frac{2}{5}$};
\node[keepstyle] (15) at (4,5) {$\frac{1}{5}$};
\node at (5,0) {$\vdots$};
\node  at (5,1) {$\frac{5}{6}$};
\node at (5,2) {$\frac{4}{6}$};
\node at (5,3) {$\frac{3}{6}$};
\node at (5,4) {$\frac{2}{6}$};
\node[keepstyle] (16) at (5,5) {$\frac{1}{6}$};
\node at (6,1) {$\cdots$};
\node at (6,2) {$\cdots$};
\node at (6,3) {$\cdots$};
\node at (6,4) {$\cdots$};
\node at (6,5) {$\cdots$};
\draw [-latex,red, thick] (11) -- (12);
\draw [-latex, red, thick] (12) -- (21);
\draw [-latex, red, thick] (21) -- (31);
\draw [-latex, red, thick] (31) -- (13);
\draw [-latex, red, thick] (13) -- (14);
\draw [-latex, red, thick] (14) -- (23);
\draw [-latex, red, thick] (23) -- (32);
\draw [-latex, red, thick] (32) -- (41);    
\draw [-latex, red, thick] (41) -- (51);
\draw [-latex, red, thick] (51) -- (15);
\draw [-latex, red, thick] (15) -- (16);
\draw [-latex, red, thick] (16) -- (25);
\draw [-latex, red, thick] (25) -- (34);
\draw [-latex, red, thick] (34) -- (43);
\draw [-latex, red, thick] (43) -- (52);
\end{tikzpicture}
\caption{Dénombrabilité de $\mathbb{Q}$}
    \label{fig:denombQ}
\end{figure}
\end{samepage}
\appendixpage\appendix
\chapter{Démonstration du théorème fondamental de l'arithmétique}\label{annexe:thmfondarith}
Pour mener à bien la démonstration, on aura besoin d'un lemme. 
\begin{lemmab}{}{ncopremsn}
  Soit $n$ un entier naturel. Alors $n$ est premier avec $n+1$ .
\end{lemmab}
\begin{proof}
La remarque précédente montre le résultat pour $n=0$. Comme 2 est premier, le résultat est vrai pour $n=1$. Supposons donc $n>1$. Soit  $k$ un entier non-nul qui divise à la fois $n~\&~n+1$, il existe alors deux entiers $l_1$ et $l_2$ tels que :  
\begin{align*}
  n+1& =kl_2=kl_{1}+1\\
\shortintertext{Soit~:} 
\cancel{k}l_2& >\cancel{k}l_1\\
\shortintertext{On a alors~:} 
1& =k(l_2-l_1)\\
\intertext{Donc $k \mid 1$, et : }
k& =1
\end{align*}
Le seul entier qui puisse diviser à la fois $n$ et $n+1$. Le lemme est ainsi démontré.
\end{proof}
\chapter{Manuel des commandes}
\begin{theoremb}{Exemple}{exem}
\(\alpha\beta\gamma ABCD=\)
\end{theoremb}
\begin{remarkb}{Remarque}{rema}
\(a\)
\end{remarkb}
\begin{noteb}{Note}{not}
$a$
\end{noteb}
\begin{defb}{Définition}{def}
$a$
\end{defb}
\begin{proof}
\proofpart{test}
\[TRUC\]
\proofpart{}
\[MACHIN\]
\proofpart*{test}
\[BIDULE\]
\end{proof}
\Cref{thm:exem}, \cref{rem:rema}, \cref{note:not} \cref{def:def} 
\deffunct{f}{E}{F}{x}{f(x)}
\backmatter
\begingroup
\raggedright
\nocite{*}
\printbibliography
\endgroup
\end{document}
%%% Local Variables:
%%% mode: latex
%%% TeX-engine: luatex
%%% TeX-master: t
%%% End:
%https://mathcs.clarku.edu/~djoyce/java/elements/bookIX/propIX20.html
